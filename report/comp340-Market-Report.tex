% Please do not change the document class
\documentclass{scrartcl}

\usepackage[hidelinks]{hyperref}
\usepackage[none]{hyphenat}
\usepackage{setspace}
\doublespace
\usepackage{amsmath}
\usepackage{graphicx}
\usepackage{wrapfig}
\graphicspath{ {./images/} }
\usepackage[title]{appendix}

% Please include a clear, concise, and descriptive title
\title{Market Report} 

\subtitle{COMP340 Market Report Report}

\author{1507516}

\begin{document}

\maketitle


\section{Career Goals}
%Illustrate the job market
%NOTES
% Talk to graduates about advice, and about the industry
%Talk to an older developer about their experiance
% 5W \& H WHY WHEN HOW ETC.
% Analyse their answeres and look at how they view the game process ( if aplicable)
% Look for similarities and differences in their answers
% Ask what kind of work they do

% Analyse at least 3 related job adverts
%Introduction
When I finish university I will plan to try and find a games studio to work for, whether that be indie or AAA. If that fails, I may start looking for an IT tech job just to get some work experience, as they seem to be more common.
I would preferably like to find a studio that makes games using C++ as that is my preferred language, however I am flexible for others that use scripting languages etc.
When it comes to what programming roles I wish to pursue after university, I am interested in procedural generation and AI, however I am open to doing generalist programming roles aswell.


\subsection{Current and future professional practice}
From my first job in the industry it would be preferable that there is a way for me to advance my skills and programming knowledge in a variety of areas, as well as learn more industry standard technologies, that will allow me to advance my career in the games industry. 
There are a few games companies that I have been following development of closely, and ultimately would love to work for, so I will be sending my job applications to them first.
A few examples of the kind of games companies I would love to work for are; Fronteir, Cloud Imperium Games, Blizzard and others.

The goal for later on in life is to have my own Indie games studio with a few of my friends who are currently in the games industry.

\subsection{Aspirations} 


\section{The job market}
This report will review requirements for three different jobs in companies I would like to apply for. From these requirements I will summarize what I have done, or what I will do to meet the goals. 


\subsection{Job Advert One}
Graduate Programmer, Frontier Developments, Cambridge  \cite{JobOne}. 
I found that Frontier was hiring from indeed, so I went to their website and found the graduate job roles that I would like to apply for.
From the advice I got from Ben Dixon, I decided to look at their
I figure that they would likely look at the applications that come through from their website rather than applications that come from indeed or other job recruitment sites.



\textbf{Requirements}
Good degree in relevant subject.
 Frontier has a page for programmer guidance, which helps outline what they would like from applicants \cite{FrontierAdvice}.

Some key points from that page are:
\begin{itemize}
	\item CV should be a doc or pdf and clearly show your education and grades attained.
	\item Use work samples to show us what sets you apart from other applicants, and show personal projects, as work from uni courses tend to be similar.
	\item Game jams typically don't give you much time to work with, so also include some bigger projects if possible. Also if any work was done in a group, explain what your role was.
	\item If you have developed games or demos then the easiest way for us to review these is a highlights video hosted online, with voice-over if you want to explain what we're looking at. Screenshots with text is also fine.
	\item Show sample source code via public source control.
\end{itemize}


\textbf{Required and Desired Skills}
Frontier uses their own game engine, developed in C++, which is the programming language of my choice.
They offer a range of specialisations for graduates to apply for, however the role I am interested in is either Engine or Generalist programmer.


\textbf{Strategy} % Highlight the trajectory of current and future professional practice
%How this leads into my future programming career
%Take away points form this job

Within my website I only highlighted my big projects, especially the projects that I worked on by myself.
Throughout my time at uni, I have checked almost every bit of work I have done into source control, so almost all the code I have done is made available. Furthermore within my website portfolio, I have linked the relevant github repo to that site, so if anyone is curios to how I implemented something, they can easily look.

From this I will create a YouTube channel that contains all the game related work I have done, especially with my own projects that I worked on over the summers at uni.








\subsection{Job Advert Two}
Junior Progammer at Ubisoft, London \cite{JobTwo}

\textbf{Link:} \url{https://www.ubisoft.com/en-US/careers/search.aspx#sr-post-id=743999665635237&is-redirect=true}

\textbf{Requirements}
\begin{itemize}
	\item Good understanding of C\# 
	\item Excellent oral and written skills in English
	\item Experience using source control software (such as Perforce/Git)
	\item Confidence with debugging in Unity
\end{itemize}

\textbf{Strategy}
This job role is looking for a junior programmer that has a good understanding of C\# and unity. 
The company \textit{Future Games} is a Ubisoft studio that develops smartphone and tablet games, that work on the Hungry Shark series.
This will be a good way to get my foot in the door to the games industry.
 









\subsection{Job Advert Three}
Games Programmer, Aardvark Swift, Cambridge

\textbf{Link:}
\url{http://gamesjobboard.jobthread.com/job/programmer-video-games-united-kingdom-aardvark-swift-11665e5feb/?d=1&source=site_search}


\textbf{Requirements}
Advanced understanding of C++ and object oriented programming
– Well organized, and capable of code design
Familirarity with PS3/Xbox Development





\subsection{Comparing the job market}




\subsection{Comparing the advice from people in the industry}
An article on Quora \cite{codeBookreview} recommends the book ``Cracking the Code Interview'' for learning how to do coding interviews, so when most of my modules have been handed in, I will start reading code interview books to get experienced with what kind of questions might come up in programmer interviews.





%Compile insights gathered by interviewing relevant experts and mentors
\section{Insights from experts and mentors}

I contacted one of my old friends, who did a computing degree at Southampton uni, and now is working at Jagex as a big data engineer, and has been for over three years. 
I contacted him asking some questions in regards to how he feels at his new job and what it is like to work in the industry.

One of his comments about how he heard about the job vacancy was he found it on their website, as he said games companies are highly sort after, and like people with a ``go do it'' attitude.
From that I will start looking at game company websites that I would like to work for, rather than looking on job sites trying to find them.

Another one of his comments stated that he applied for a graduate position, which meaned the company put aside time and resources in order for him to patch the gaps in the knowledge he wasn't taught in university.



\section{Conclusion}

\begin{appendices}
\section{Ben Dixon - Jagex}
Job: Big Data Engineer
Company: Jagex Game Studio
Was there anything that surprised you, either good or bad about working the game industry?
Mainly around how much goes into not just building, but maintaining an MMO! In other AAA titles, a lot of time and effort goes into designing and building the game and then after the game’s release it is vastly reduced. With MMOs, they are constantly improving the engine, adding new content, fixing bugs on a weekly basis.
And then on top of this, there are whole chunks of the business that you just don’t even think about that are needed to keep the game going. IT teams maintaining and improving the servers and architecture, customer support for handling account queries and moderating to protect players (particularly younger ones), community management to communicate with the players, gather feedback and inform players about updates and changes… the list goes on.    
\par
\textbf{What is it like working in a team?}
\par
Working in teams to build games is generally an awesome experience. Within a games company, or any company for that matter, you will be working with genuinely skilled people (they have been hired for these skills!), often with many years experience in the industry. It’s really inspiring for me, to be able to work alongside them.
You must learn to compromise though! When you get really into working on something, be it the art, design, development, your creative side will really get into the flow and you’ll start generating all sorts of great ideas about what you could do next or how you could improve the project in some way. The first thing to remember is, they are just ideas and not everyone will agree with you. Everyone’s vision of the project can be different and they will all have their own ideas too. Finding alternate solutions or compromises is vital to keeping the team together and delivering it on time. Which brings me to the second thing to remember; that you are working for a company (in my case of 300+ people) where some things may simply be out of your control, no matter how vocal you are. Some things have a reason for being the way they are, be it good or bad, and you may have to either accept it or move on.
\par
\textbf{How do you feel about the size of your team? Does the work you do feel rewarding?}
\par
I work in a very small team of 2 but also closely with our grander department of 15 people and several others across the company. It’s great to be able to have some freedom and breadth with what I do; as there is only 2 of us and a large domain of work to cover, I get the chance to conceive, plan and produce entire projects myself. It’s a great learning experience for my career and it’s really rewarding to come in every day, see the system I made doing its thing and say, “I built that!”
Working with so many other departments across the company is also pretty cool. I get to see and be involved in a lot more of what goes on than a lot of people normally would. 
\par
\textbf{How easy was it to adapt and how helpful were people when you started?}
\par
So when I started, there was a lot of domain knowledge, specific to my job, which I just didn’t have (mainly because there were no modules on it at Uni). But as I was joining in a graduate role, they were well prepared for this and helped me patch up some of the gaps in my understanding, as well as introducing me to entirely new pieces of software. I wouldn’t say it was easy, it got pretty technical at times, but I like a challenge and I enjoyed it. I would highly recommend going for graduate positions where available as they are likely to have time and money set aside for training you. 
And yes, everyone was helpful!
\par
\textbf{What was the interview process like?}
\par
My interview started with a tour around the studio! It was amazing to see; for the first time I got a glimpse of what it was like to work in a games company and what my life could be. We did finally get around to a more formal interview, with 2 people, which consisted of a beginning chat about my background and then led into some technical coding and planning tasks, and ending with some discussion on conflict resolution and team working. 
The feeling throughout was fairly informal though, with a lot of back and forth discussion rather than just question and answer. I think their first question was, “So what games have you been playing recently?” which led us into a conversation about games we played as kids.
\par
\textbf{How did you first hear about the job vacancy?}
\par
I found it on their website. Jobs at games companies seem to be highly sort after and they are looking for people with a “go do it” attitude. I think you’ll rarely find them using agencies for hiring; they are looking for people to come to them.
\par
\textbf{Do you feel you are given the chance to progress within the company?}
\par
Yes, they even have an entire department dedicated to it (Learning \& Development)! They provide ample resources, from online training courses to knowledge sharing seminars and funding to go to events that will help you achieve your goals. Along with opportunities for promotion, there is also an internal job vacancy board which jobs will often appear on first to give you the chance to move about/up if you ever feel like your interests are changing or that you may be better suited in another role.


\section{Appendix item two}
\end{appendices}

\bibliographystyle{ieeetr}
\bibliography{comp340-Market-Report}
\end{document}
